\section{System  and data  model}
\label{sec:overview}

\specula is a data-store designed to operate in large-scale, geo-distributed data centers. Our current implementation of \specula runs on a key-value store, although the protocol it employs is agnostic to the underlying data model (e.g., relational). The whole dataset is partitioned across multiple replicated shards, or partitions, which are maintained by servers (or nodes) scattered across geographically dispersed data centers.

Each data partition is responsible for a key range and it maintains multiple timestamped versions for each key.  Synchronous master-slave replication is employed to enforce fault tolerance and transparent failover, as used in \cite{spanner, baker2011megastore}. A partition has a master replica and several slave replicas, and, at commit time, update transactions contact the masters of the partitions they accessed. These verify whether transactions can be correctly serialized  and propagate their updates, along with any metadata (e.g., locks held by the transaction) required for their recovery, to its replicas.
This scheme allows for transparent fail-over, in case of failure of master replicas. Further, it allows reads to be served by any replica, which allows client to freely select the one that is geographically closer.


Analogously to other recent geo-distributed data stores~\cite{spanner, clocksi, terry2013consistency}, \specula supports flexible data placement policies. A server can host multiple data partitions, either master replicas or slave replicas. A data partition can be replicated within a datacenter and across datacenters, and no server or datacenter is required to host all partitions. 

Last but not least, each server is equipped with a standard hardware clock, which we only assume to monotonically move forward.

\section{Programming Model}
\label{sec:programming_model}

ACID transactions provide a powerful programming abstraction by ensuring Atomicity, Consistency, Isolation and Durability. To enforce these properties, ACID transactions specify a set of rigorous execution rules, which often require synchronization between geo-dispersed data replicas. In state of the art data stores, like Google's Spanner~\cite{spanner}, committing a transaction entails the execution of a two phase commit protocol among the master of different partitions along with a consensus instance among the replicas of each partition~\cite{cockroach,spanner,schiper2010p}. These protocols introduce in the critical path of the commit phase the latency of several inter-datacenters communication steps, which is detrimental to user experience \cite{schurman2009user}.

An intuitive way to mask this large inter-datacenters latency is to make an optimistic guess about the result of a transaction, instead of waiting out the uncertainty. The consequence is that, rather than always getting correct and irrevocable results, clients get predicted results that may occasionally go wrong. As we have discussed in \S \ref{sec:introduction}, although replying users with a probably correct answer, instead of a definitive one, seems unacceptable in an idealistic business model, it is indeed commonly used in practice \cite{helland2009building, brewer2012cap}, to tame large latency and low availability: an ATM allows clients to withdraw money even in the presence of network partition, and charges users overdraft fees if necessary; flight companies usually oversell tickets to ensure high occupancy rate, to achieve higher revenue while risking compensating passengers that could not get booked seats.

Allowing speculative commit raises a relevant question: if clients try to read pending updates of speculatively-committed transactions, how should the system react? A straw-man proposal is to ignore these speculative versions (by returning the last committed version or blocking clients until these versions are committed or aborted), therefore transactions still execute in a strongly-consistent fashion and no extra complexity is introduced (as exploited by \cite{pang2014planet}). Albeit neat, this approach has potential problems: returning older versions gives clients weird semantics, while blocking until speculative updated are finalized can introduce large latency especially when the system has hotspots. To avoid these problems, \specula allows speculatively-committed transactions' updates to be visible to others transactions. In further sections, we shall more precisely define the consistency semantics (\S \ref{subsec:ssi}) as well as how the protocol design.

\iffalse
The prevalence of geo-replication has fueled the Eventual Consistency (BASE transactions) v.s. Strong Consistency (ACID transactions) argument. While achieving high availability and low latency by avoiding inter-datacenter latency, eventual consistency only guarantees that ``...changes made to one copy eventually migrate to all", according to one of its earliest definitions \cite{kawell1988replicated}. Programmers have to explicitly code ad-hoc logic to deal with concurrency anomalies and ensure consistency requirements. On the other hand, strong consistency provides a powerful programming abstraction via enforcing a set of rigorous execution rules, which often requires synchronization between geo-dispersed data replicas. In state of the art data stores, like Google's Spanner~\cite{spanner}, committing a transaction entails the execution of a two phase commit protocol among the master of different partitions along with a consensus instance among the replicas of each partition~\cite{cockroach,spanner,schiper2010p}. These protocols introduce in the critical path of the commit phase the latency of several inter-datacenters communication steps. Such a latency can not only have a detrimental impact on user experience in human-facing interactions~\cite{schurman2009user}, but also lead to lock convoying effects that can severely hinder concurrency and throughput.
\fi

We illustrate how programmers can exert control over speculation via \specula's  programming interface true the example code of a simplified online shopping transaction (see Listing 1), . %Based on a programming interface as a conventional transactional API that allows issuing read/write on key-value pairs, we extend its transactional demarcation interface with speculative options. Besides allowing speculative read/commit, the interface also allows registering callbacks that define what to perform when a transaction reaches speculative commit and final commit/abort states.


%In order to allow programmers to  control these factors, \specula exposes an intuitive programming interface, which extends in a simple way the classic API used by conventional transactional systems. 


%Figure~\ref{lst:interface}.

\begin{lstlisting}[caption=Example transaction that buys an item]   
checkRisk(ItemInfo, TxInfo) { 
    return ItemInfo.Price < 100 &&
                    TxInfo.CommitProb > 0.9
}

scHandler(TxInfo) {
    // Display "Your order has been placed."
}

fcHandler(TxInfo) {
    // Send an email to the user notifying success order
}

faHandler(TxInfo) {
    // Retry three times, if still fails then,
    // Send an email to the user saying sorry
}

buyItemTx(ItemKey, ItemInfo, TxInfo){
    Tx = beginTx(
                checkRisk(ItemInfo, TxInfo),//If speculative commit
                True,//Allow speculative read
                scHandler(TxInfo),//Speculative commit callback
                fcHandler(TxInfo),//Final commit callback
                faHandler(TxInfo));//Final abort callback
    Item = tx.read(ItemKey);
    Item.quantity -= 1;
    Tx.write(ItemKey, Item);
    Tx.commit();
}
\end{lstlisting}

{\bf Enabling speculative commits and/or reads.} \specula extends the transaction begin API with two optional boolean parameters, namely \textit{specCommit} and \textit{specRead} (which default to true if they are left unspecified). The former allows programmers to define whether a transaction should be speculatively committed: as in the example, programmers specify a \textit{checkRisk} function to decide whether to perform speculative commit. We shall define more rigorously the consistency guarantees ensured by \specula for a speculatively committed transaction in the next section. 
\iffalse
Yet, in a nutshell, a speculatively committed transaction is guaranteed to have executed respecting the SI semantics for a history comprising all final committed transactions and locally originated speculatively committed transactions.
\fi

The \textit{specRead} parameter allows to define whether a transaction may observe data items versions created by previously speculatively committed transactions, or whether it should rather be exclusively allowed to read final committed data item versions.

As illustrated in the code example, the use of these two flags empowers application (or middleware) programmers to craft logic aimed to maximize the chance of successful speculation, as well as to circumscribe the scenarios in which speculation should be used. For instance, in the considered example, a transaction can be allowed to be speculatively committed only if it is buying an inexpensive item and if it has experienced a low abort rate over a recent time-window, and is allowed to read speculatively committed data versions. Yet, the two flags can be used independently. For instance, a developer may decide to allow a transaction $T$ to observe speculatively committed data versions even though $T$ is not allowed to speculative commit. This may enhance the chances of having $T$ commit, by ensuring that $T$ can read the versions generated by concurrent speculatively committed transactions, while avoiding to ever expose speculative state to the client that issued $T$.

It should be noted that enabling the possibility of observing speculatively committed data version raises subtle issues.  Snapshot Isolation (SI) has not been defined with speculation in mind. In particular, it does not specify which data snapshots should be observable in histories encompassing speculatively committed transactions. We cope with this issue in the following Section, in which we shall introduce a generalization of SI, called SPeculative Snapshot Isolation (SPSI), that will serve as target consistency criterion for \specula.


{\bf Reacting to Speculative Commit and Final Commit/Abort Events.} Upon beginning a new transaction, programmers have to specify three callback functions: \textit{onSpeculativeCommit()}, \textit{onFinalAbort()} and \textit{onFinalCommit()}.

As its name suggests, onSpeculativeCommit() allows programmers to define which code should be executed if a transaction is speculatively committed (i.e., it passes local certification). Furthermore, the other two callbacks (\textit{onFinalAbort()} and \textit{onFinalCommit()} ) specify the actions to take when a transaction is eventually committed or aborted after speculative commit, respectively. The example code presents a simple way to avoid notifying users too often about speculative abort: since the semantics of buying an items does not change during different runs, the system can just retry the buy item transaction if it gets aborted after speculation.

\section{Consistency semantics}
By introducing speculative commit and speculative read, programmers gain flexible control over the execution of transactions to mitigate large latency and improve performance. Nevertheless, developers may have perplexity about their exact consistency semantics. Before proposing the supported consistency criteria, namely SPeculative Snapshot Isolation (SPSI), we reason about the properties that leads to the design of SPSI, which aims at providing meaningful and intuitive semantics to programmers despite using these speculative techniques.

First, despite the existence of speculative transactions, all finally committed transactions should still preserve the original guaranteed strong consistency criteria, e.g. no write-write conflict among concurrent transactions as in Snapshot Isolation. In fact, a simple approach to reduce synchronization is to allow executing and committing transactions with different consistency levels, e.g. executing certain `important' transactions with Serializability while other `non-important' ones with weak isolation levels, such as Read Committed. However, mixing consistency indeed degrades the consistency guarantee of the system to the weakest one supported, as strict execution rules enforced by a consistency level can be violated by transactions of lower consistency semantics. This can easily introduce concurrent anomalies not conceived by application developers.

Secondly, if a transaction is not supposed to read speculative data, it should observe a snapshot as it would read in a strongly-consistent transactional protocol. This guarantee ensures that making one transaction speculative will not require reasoning or rewriting any other transactions.

Last but not least, we would like `speculative snapshots' read by transactions to give meaningful semantics. The most `meaningful' snapshots would be ones that do not contain states that could not exist in the original strong consistency level, such as conflicting writes, non atomic updates etc. Providing meaningful semantics alleviates programmers' effort to write applications in a speculative fashion, as they only need to think about when to perform speculative operations and compensation logics when speculative transactions abort, but not dealing with any concurrency anomalies. 

\subsection{Speculative Snapshot Isolation}
\label{subsec:ssi}
As mentioned above, SPSI, namely the target consistency criterion for \specula, has been designed to generalize the well-known, and widely adopted, SI, whose definition we recall in the following~\cite{weikum2001transactional}:

\begin{itemize}
\item \textsc{SI Property 1.} (\textit{Snapshot Read}) All operations read the most
recent committed version as of the time when the transaction began.
\item \textsc{SI Property 2.} (\textit{No Write-Write Conflicts}) The write sets of
each pair of committed concurrent transactions must be disjoint.
\end{itemize}

\iffalse
The design process that led us to the definition of SPSI has been driven by three main goals/requirements:
\begin{itemize}
\item providing meaningful correctness guarantees in presence of executions encompassing both speculatively and final committed transactions;
\item preserving the intuitiveness and simplicity of the original SI specification;
\item enabling highly scalable and efficient implementations, in which transactions can be speculatively committed after a local certification phase.

Guided by these three objectives,
\end{itemize}
\fi

We specify SPSI via the following properties:
\begin{itemize}
\item \textsc{SPSI Property 1.} (\textit{Speculative Snapshot Read}) A transaction $T$ originated at a node $S$ must observe the most recent data item versions created by  transactions that i) had final committed by the time $T$ started (independently of the site where these transactions originated), or ii) had speculatively committed by the time T started and were also originated at  $S$.
\item \textsc{SPSI Property 2.} (\textit{No Write-Write Conflicts among Final Committed Transactions}) The write sets of
each pair of finally committed concurrent transactions must be disjoint.
\item \textsc{SPSI Property 3.} (\textit{No Write-Write Conflicts among Speculatively Committed and Committed Transactions}) The write sets of a speculatively committed originated at a node $n$ must not intersect with the write-set of concurrent transaction that i) has been final committed, or ii) has been speculatively committed and originated at node $n$.
\item \textsc{SPSI Property 4.} (\textit{No Dependencies from Aborted Transactions}) A transaction can only be final committed if it does not depend on a speculatively committed or aborted transaction.
\end{itemize}

SPSI Property 1 extends the notion of snapshot, at the basis of the SI definition, to include among the set of  transactions visible to a transaction $T$ originated on a node S at time $t$ all the transactions that have also originated on $S$ and that were speculatively committed by time $T$. Notice that this guarantee applies to all transactions, including those that have to be eventually aborted due to conflicts with either local or remote transactions. This property provides the illusion that a transaction $T$ executes on an immutable snapshot, which reflects the execution of all the transactions speculatively committed before $T$'s activation, and originated on the same local node. These include not only the transactions originated by the same client and speculatively committed before activating the current transaction --- from which we say that the current transaction has developed a \textit{flow dependence} - but also transactions from which the current transaction may have already have developed a \textit{data dependence} by observing their speculatively committed versions during one of of its previous read operations. 

SPSI Property 2 coincides with SI's property 2, ensuring the same semantics regarding the absence of write-write conflicts among concurrent final committed transactions. SPSI Property 3 expresses analogous guarantees for speculatively committed transactions, although with a relevant difference: such guarantees are restricted only to pairs of transactions that are originated at the same node. It is worth highlighting that the choice of restricting, in SPSI Properties 1 and 3, the guarantees regarding speculative committed versions only to transactions originated at the same site is fundamental to allow implementing efficient schemes that can decide whether to speculative commit a transaction based solely on local knowledge (i.e., avoiding expensive remote certification phases).

Finally, SPSI Property 4 ensures that a transaction can be final committed only if it does not have any dependence on transactions that are not, in their turn, final committed --- which guarantees that all possible causal dependencies (i.e., flow and data dependences) that a transaction may have developed are towards transactions whose outcome is not or can never become an abort.

Overall, SPSI Properties 1, 3 and 4 greatly restrict the spectrum of anomalies that can be experienced by speculatively committed transactions, by limiting them to conflicts developed with concurrent transactions originated at remote sites. More formally, those properties ensure that applications that expose/observe speculative states always produce/execute on snapshots  that are correct if one applies the SI criterion to the transaction history comprising the transactions which were known to the node in which such applications executed at the time in which they started executing a transaction. 

