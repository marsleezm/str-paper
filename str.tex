% This is "sig-alternate.tex" V2.1 April 2013
% This file should be compiled with V2.5 of "sig-alternate.cls" May 2012
%
% This example file demonstrates the use of the 'sig-alternate.cls'
% V2.5 LaTeX2e document class file. It is for those submitting
% articles to ACM Conference Proceedings WHO DO NOT WISH TO
% STRICTLY ADHERE TO THE SIGS (PUBS-BOARD-ENDORSED) STYLE.
% The 'sig-alternate.cls' file will produce a similar-looking,
% albeit, 'tighter' paper resulting in, invariably, fewer pages.
%
% ----------------------------------------------------------------------------------------------------------------
% This .tex file (and associated .cls V2.5) produces:
%       1) The Permission Statement
%       2) The Conference (location) Info information
%       3) The Copyright Line with ACM data
%       4) NO page numbers
%
% as against the acm_proc_article-sp.cls file which
% DOES NOT produce 1) thru' 3) above.
%
% Using 'sig-alternate.cls' you have control, however, from within
% the source .tex file, over both the CopyrightYear
% (defaulted to 200X) and the ACM Copyright Data
% (defaulted to X-XXXXX-XX-X/XX/XX).
% e.g.
% \CopyrightYear{2007} will cause 2007 to appear in the copyright line.
% \crdata{0-12345-67-8/90/12} will cause 0-12345-67-8/90/12 to appear in the copyright line.
%
% ---------------------------------------------------------------------------------------------------------------
% This .tex source is an example which *does* use
% the .bib file (from which the .bbl file % is produced).
% REMEMBER HOWEVER: After having produced the .bbl file,
% and prior to final submission, you *NEED* to 'insert'
% your .bbl file into your source .tex file so as to provide
% ONE 'self-contained' source file.
%
% ================= IF YOU HAVE QUESTIONS =======================
% Questions regarding the SIGS styles, SIGS policies and
% procedures, Conferences etc. should be sent to
% Adrienne Griscti (griscti@acm.org)
%
% Technical questions _only_ to
% Gerald Murray (murray@hq.acm.org)
% ===============================================================
%
% For tracking purposes - this is V2.0 - May 2012

\documentclass[preprint]{sigplanconf-eurosys}

%\usepackage{amsmath}
\usepackage{courier}
\usepackage{listings}
\usepackage{amssymb}
%\usepackage{amsthm}
\usepackage{xcolor}
\usepackage{subcaption}
\usepackage{tabularx}
\usepackage{diagbox}
\usepackage{hyperref}
\usepackage[english,ruled]{algorithm2e}
\usepackage{multirow}
\usepackage{float}
\usepackage{color}



\newcommand*{\ttfamilywithbold}{\fontfamily{lmtt}\selectfont}

\definecolor{mygreen}{rgb}{0,0.6,0}
\definecolor{mygray}{rgb}{0.5,0.5,0.5}
\definecolor{mymauve}{rgb}{0.58,0,0.82}
\lstset{ %
  %emph={baz},
  %emphstyle=\textbf
  basicstyle=\scriptsize\ttfamily,        % the size of the fonts that are used for the code
  captionpos=b,                    % sets the caption-position to bottom
  commentstyle=\color{mygreen},    % comment style
  escapeinside={\%*}{*)},          % if you want to add LaTeX within your code
  showspaces=false,
  showstringspaces=false,
  framerule=1pt,
  frame=bt,	                   % adds a frame around the code
  keepspaces=true,                 % keeps spaces in text, useful for keeping indentation of code (possibly needs columns=flexible)
  keywordstyle=\bfseries,,       % keyword style
  language=Java,                 % the language of the code
  rulecolor=\color{black},         % if not set, the frame-color may be changed on line-breaks within not-black text (e.g. comments (green here))
  stringstyle=\color{mymauve},     % string literal style
  tabsize=2,	                   % sets default tabsize to 2 spaces
  title=\lstname                   % show the filename of files included with \lstinputlisting; also try caption instead of title
}



\begin{document}

\newcommand\VRule[1][\arrayrulewidth]{\vrule width #1}
\newcommand{\specula}{STR\xspace}

\title{STR: speculative execution for partially-replicated datastores}

\authorinfo{paper number XXX of XXX}
%\authorinfo{Zhongmiao Li$^{\dagger}$$^{\star}$, Peter Van Roy$^{\dagger}$ and Paolo Romano$^{\star}$ \\
%	 {\normalsize $^ {\dagger}$Universit\'e catholique de Louvain \quad
%          $^{\star}$Instituto Superior T\'ecnico} \\ [2mm]
%          \email{\{zhongmiao.li, peter.vanroy\}@uclouvain.be}\\
%          \email{\{zhongmiao.li, paolo.romano\}@ist.utl.pt}
%}


% There's nothing stopping you putting the seventh, eighth, etc.
% author on the opening page (as the 'third row') but we ask,
% for aesthetic reasons that you place these 'additional authors'
% in the \additional authors block, viz.

% Just remember to make sure that the TOTAL number of authors
% is the number that will appear on the first page PLUS the
% number that will appear in the \additionalauthors section.

\maketitle
\begin{abstract}

%The performance of geographically distributed data stores is challenged by the existence of ineluctably large communication delays between datacenters, which exacerbates the costs induced to ensure consistency of replicated data.
 This article introduces \specula (Speculative Transactional Replication), an innovative geo-replicated transactional data-store that exploits speculative concurrency control techniques to mask the latency of inter data-center synchronization. \specula provides programmers with the flexibility to adopt, on a per transaction basis, either strong or weak consistency semantics.

On the one hand, the data replication protocol employed by \specula to ensure strong consistency, i.e., Snapshot Isolation (SI), departs from conventional designs, which  conservatively block transactions whenever these attempt to access data  updated by concurrently executing transactions. Conversely, \specula adopts a speculative approach, which avoids blocking transactions by allowing them to selectively observe the data-item versions produced by concurrent (pre-committed) transactions. This design choice allows \specula to achieve  up to XX$\times$ higher throughput and up to YY$\times$ lower latency than in state of the art strongly-consistent data stores.%, which suggests that, in geo-replicated datastores, the advantages stemming from embracing the risk of cascading aborts largely outweigh its drawbacks.

On the other hand, \specula's weak consistency model, SPeculative Snapshot Isolation (SPSI), allow programmers to externalize the results of a transaction while this is still undergoing its final commit/certification phase, hence allowing for further enhancing  throughput and latency. A key distinguishing treat of SPSI is that it provides clear and stringent guarantees on the atomicity and isolation of the snapshots observed and produced during transaction execution. This spares programmers from having to cope with complex concurrency bugs that could arise when using alternative weakly consistent models, like eventual consistency.
\end{abstract}


%
% The code below should be generated by the tool at
% http://dl.acm.org/ccs.cfm
% Please copy and paste the code instead of the example below. 
%

% We no longer use \terms command
%\terms{Theory}

\keywords{Speculative execution; geo-replication; transactional datastore}


\section{Conclusion}
This paper proposed {\specula}, a novel protocol that uses speculative techniques to improve performance of distributed transactions. The three novel speculative techniques, i.e. speculative commit, speculative read and precise clock, collectively mitigate the performance bottleneck of existing transactional protocol. Extensive benchmark showed that our protocol achieves significantly higher throughput than state of the art non-speculative system and is particularly beneficial when workloads have high local conflict and low remote conflict.


% We recommend abbrvnat bibliography style.
\bibliographystyle{abbrv}
\bibliography{references}
\end{document}
